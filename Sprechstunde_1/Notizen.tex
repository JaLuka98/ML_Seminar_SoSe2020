\PassOptionsToPackage{unicode}{hyperref}
\documentclass[aspectratio=1610, 9pt]{beamer}

% Load packages you need here
%\usepackage{polyglossia}
%\setmainlanguage{german}

\usepackage{csquotes}

\usepackage{amsmath}
\usepackage{amssymb}
\usepackage{mathtools}

\usepackage{braket}
\usepackage{graphicx}

\usepackage{hyperref}
\hypersetup{
  linkcolor= {tudark}, % internal links
  citecolor={tugreen}, % citations
  urlcolor={tudark} % external links/urls
  }
\usepackage{bookmark}

\usepackage[english]{babel}
\usepackage[
backend=biber,
style=authoryear-comp
]{biblatex}

\bibliography{lit.bib}

\usepackage{siunitx}
\usepackage{multicol}

\usepackage{booktabs}

\definecolor{light-gray}{HTML}{b0b5b0}

% load the theme after all packages

\usetheme[
  showtotalframes, % show total number of frames in the footline
]{tudo}

% Put settings here, like
\unimathsetup{
  math-style=ISO,
  bold-style=ISO,
  nabla=upright,
  partial=upright,
  mathrm=sym,
}

\title{Sprechstunde zu:\\Klassifikation von Tiergesichtern in drei Kategorien}
\author[T.~Magorsch,~J.~L.~Späh]{Tom Magorsch\\ Jan Lukas Späh}
\institute[ML-Seminar]{\\[0.3cm]TU Dortmund \\ \Large ML-Seminar}

\begin{document}

\maketitle


\begin{frame}{Erinnerung: Fragestellung und Datensatz}

  \begin{columns}
    \column{0.8\textwidth}
    \begin{itemize}
      \item Klassfikationsproblem: Hund, Katze, Wildtier\\
      \item $16130$ farbige Bilder verschiedener Tiere: $512\times 512$ Pixel
    \end{itemize}
    \column{0.2\textwidth}
    \includegraphics[scale=0.13]{images/cat.jpg}\\
    \includegraphics[scale=0.13]{images/dog.jpg}\\
    \includegraphics[scale=0.13]{images/wildlife.jpg}\\
  \end{columns}

\end{frame}

\begin{frame}{Training der Daten}
  \begin{itemize}
  \item \textbf{Problem:} Rohdaten passen nicht in den Arbeitsspeicher.
  \item \textbf{Lösung:} Lade Daten während dem Training in Batches in den RAM
    \begin{enumerate}
    \item Lade Bilder mit imread während dem Training
    \item Erstelle .h5 und lade Bilder daraus
    \end{enumerate}
  \item Training mit Data Generator jedoch langsamer
  \item Test mit 1200 training und 300 validation samples - 50 Mio Parameter:
    \begin{enumerate}
    \item Training mit Bildern im RAM: 261s
    \item Training mit imgrea: 338s
    \item Training mit h5: 318s
    \end{enumerate}
  \end{itemize}
\end{frame}


\begin{frame}{DNN vs CNN}
  \begin{itemize}
  \item Zwei Netzwerke die prinzipiell funktionieren.
  \item Nächster Schritt: Hyperparameter Optimierung
  \end{itemize}
  \includegraphics[scale=0.3]{images/cnn_acc.png}
  \includegraphics[scale=0.3]{images/dnn_acc.png}
  \begin{itemize}
  \item Fragestellung: Lohnt sich die volle Auflösung zu nutzen?
  \end{itemize}
\end{frame}

\begin{frame}{Alternativmethode}

  \begin{itemize}
    \item Wir brauchen eine einfache Alternativmethode als Benchmark/Motivation für anspruchsvollere Methoden (CNN,\ldots)
    \item Erste Idee: kNN-Classifier mit reinen Pixelwerten
    \begin{itemize}
      \item Vorteil: Einfach und schafft Klassfikation mit ca. 50\% Accuracy
      \item Nachteil: Speicherintensiv. Muss im Prinzip alle Trainingsdaten im RAM haben
    \end{itemize}
    \item Wie den Nachteil lösen oder umgehen?? Ideen:
    \begin{itemize}
      \item Bilder iterativ downscalen und damit arbeiten
      \item Datensatz aufteilen in $n$ Häppchen, die in RAM passen, kNN fitten und evaluieren mit angemessen val\_split\\
      \rightarrow{} Erhalte val\_acc auf komplettem Datensatz mit Unsicherheit
    \end{itemize}
    \item Zweite Idee: PCA $\to$ SVM, BDT oder Ähnliches
    \item Es existieren Methoden, PCA inkrementell zu machen. Alternativ: PCA berechnen in Colab, abspeichern und laden
  \end{itemize}

\end{frame}


\begin{frame}{Ideen zum Ausbau des Projekts}

  \begin{itemize}
    \item Zusammenfassend: Erste Alternativmethode funktioniert schlechter als bereits zufriedenstellendes CNN
    \item Wie weiter? Ist das sinnvoll?
    \begin{itemize}
      \item Hyperparametersuche für kNN, Alternativmethoden und Netze
      \item Bestes Netz aufschneiden und eine Lage clustern\\
      \rightarrow{} Von Hand inspizieren. Sind Tierarten oder Muster erkennbar?
      \item Einfall: Klassischen Hund/Katze Datensatz und Performance testen.
      \item Auch möglich, um Modell zu interpretieren: Selbst einige wenige Bilder mit bekannten Arten heraussuchen und schauen, was passiert.
    \end{itemize}
    \item Einschätzung von Euch: Guter Plan? Ab wo ausreichend für Projekt?
  \end{itemize}

\end{frame}



\end{document}
