\PassOptionsToPackage{unicode}{hyperref}
\documentclass[aspectratio=1610, 9pt]{beamer}

% Load packages you need here
%\usepackage{polyglossia}
%\setmainlanguage{german}

\usepackage{csquotes}

\usepackage{subfigure}

\usepackage{amsmath}
\usepackage{amssymb}
\usepackage{mathtools}

\usepackage{braket}
\usepackage{graphicx}

\usepackage{hyperref}
\hypersetup{
  linkcolor= {tudark}, % internal links
  citecolor={tugreen}, % citations
  urlcolor={tudark} % external links/urls
  }
\usepackage{bookmark}

\usepackage[english]{babel}
\usepackage[
backend=biber,
style=authoryear-comp
]{biblatex}

\bibliography{lit.bib}

\usepackage{siunitx}
\usepackage{multicol}

\usepackage{booktabs}

\definecolor{light-gray}{HTML}{b0b5b0}

% load the theme after all packages

\usetheme[
  showtotalframes, % show total number of frames in the footline
]{tudo}

% Put settings here, like
\unimathsetup{
  math-style=ISO,
  bold-style=ISO,
  nabla=upright,
  partial=upright,
  mathrm=sym,
}

\title{Klassifikation von Tiergesichtern in drei Kategorien}
\author[T.~Magorsch,~J.~L.~Späh]{Tom Magorsch\\ Jan Lukas Späh}
\institute[ML-Seminar]{\\[0.3cm]TU Dortmund \\ \Large ML-Seminar}

\begin{document}



\maketitle

\begin{frame}{Datensatz und Fragestellung}
  \begin{columns}

    \column{0.8\textwidth}

    \begin{itemize}
    \item Quelle: \href{https://www.kaggle.com/andrewmvd/animal-faces?}{Kaggle}, Datensatz aus Dezember 2019 (\href{https://arxiv.org/abs/1912.01865}{arxiv:1912.01865}) mit \href{https://github.com/clovaai/stargan-v2}{Git-Repo}
    \item Lizenz: CreativeCommons (CC BY-NC 4.0)
    \item $16136$ Farbbilder verschiedener Tiere: $512\times 512$ Pixel
    \item Bilder in drei Kategorien aufgeteilt: (jeweils ca. 5000 Bilder)
      \begin{enumerate}
      \item Cat
      \item Dog
      \item Wildlife
      \end{enumerate}
    \item Datensatz bereits aufgeteilt in train (14633) und test (1503)
    \item Fragestellung der Klassifikation:\\
    \rightarrow{} ``Befindet sich auf dem Bild ein Hund, eine Katze oder etwas anderes?''
    \end{itemize}

    \column{0.2\textwidth}
    \includegraphics[scale=0.13]{images/cat.jpg}\\
    \includegraphics[scale=0.13]{images/dog.jpg}\\
    \includegraphics[scale=0.13]{images/wildlife.jpg}\\

  \end{columns}
\end{frame}

\begin{frame}{Alternativmethode}

  \begin{itemize}
    \item Methoden ohne Deep-Learning können meist schlecht mit reinen Pixeldaten umgehen\\
    \rightarrow{} Hochaufgelöster Datensatz: Bilder passen nicht ansatzweise in den RAM
    \begin{enumerate}
      \item 3D Colour Histograms: $512\times512\times3=786\,432 \to 8\times8\times8=512$ features
      \item PCA, die $95\%$ der Varianz erhält $\to 112$ features
      \item kNN-Klassifikation: Acc., Prec. und Rec. von $51$ bis $55$ Prozent
    \end{enumerate}
  \end{itemize}

  \begin{figure}
	   \centering
	   \subfigure{\label{fig:a}\includegraphics[width=60mm]{figures/kde_sns.pdf}}
	   \hspace{1cm}
	   \subfigure{\label{fig:b}\includegraphics[width=60mm]{figures/kde_sns_2.pdf}}
  \end{figure}

\end{frame}

\begin{frame}{Optimierung zweier CNN}
  ???
\end{frame}


%\begin{frame}{Methode zur Klassifikation und Vergleich mit Alternativen}
%  \begin{itemize}
%    \item Einfache Algorithmen sind kaum in der Lage, Bilder mit vergleichbarer Genauigkeit wie Deep-Learning-Methoden zu klassifizieren\\
%    \rightarrow{} Trainiere unterschiedliche neuronale Netze
%    \item Beispiel: Flaches NN, Tiefes NN und CNN
%    \item Mögliche einfache Vergleichsmethode für Bildklassifikation: z.B. kNN:
%    \begin{enumerate}
%      \item Extrahiere Features manuell aus den Bilder (z. B. Farbhistogram, Kontrast, Form-Features)
%      \item Definiere Metrik über dem Feature-Raum
%      \item Nutze Metrik für klassischen kNN
%    \end{enumerate}
%    \item Performance Measure für GAN: Kann ein DNN die generierten von den echten Bildern unterscheiden ?
%  \end{itemize}
%
%\end{frame}



\end{document}
